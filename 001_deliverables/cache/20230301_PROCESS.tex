% Options for packages loaded elsewhere
\PassOptionsToPackage{unicode}{hyperref}
\PassOptionsToPackage{hyphens}{url}
%
\documentclass[
]{article}
\usepackage{amsmath,amssymb}
\usepackage{lmodern}
\usepackage{iftex}
\ifPDFTeX
  \usepackage[T1]{fontenc}
  \usepackage[utf8]{inputenc}
  \usepackage{textcomp} % provide euro and other symbols
\else % if luatex or xetex
  \usepackage{unicode-math}
  \defaultfontfeatures{Scale=MatchLowercase}
  \defaultfontfeatures[\rmfamily]{Ligatures=TeX,Scale=1}
  \setmainfont[]{Bahnschrift:style=Bold}
  \setmonofont[]{Bahnschrift:style=Condensed}
\fi
% Use upquote if available, for straight quotes in verbatim environments
\IfFileExists{upquote.sty}{\usepackage{upquote}}{}
\IfFileExists{microtype.sty}{% use microtype if available
  \usepackage[]{microtype}
  \UseMicrotypeSet[protrusion]{basicmath} % disable protrusion for tt fonts
}{}
\makeatletter
\@ifundefined{KOMAClassName}{% if non-KOMA class
  \IfFileExists{parskip.sty}{%
    \usepackage{parskip}
  }{% else
    \setlength{\parindent}{0pt}
    \setlength{\parskip}{6pt plus 2pt minus 1pt}}
}{% if KOMA class
  \KOMAoptions{parskip=half}}
\makeatother
\usepackage{xcolor}
\usepackage[margin=1in]{geometry}
\usepackage{color}
\usepackage{fancyvrb}
\newcommand{\VerbBar}{|}
\newcommand{\VERB}{\Verb[commandchars=\\\{\}]}
\DefineVerbatimEnvironment{Highlighting}{Verbatim}{commandchars=\\\{\}}
% Add ',fontsize=\small' for more characters per line
\usepackage{framed}
\definecolor{shadecolor}{RGB}{248,248,248}
\newenvironment{Shaded}{\begin{snugshade}}{\end{snugshade}}
\newcommand{\AlertTok}[1]{\textcolor[rgb]{0.94,0.16,0.16}{#1}}
\newcommand{\AnnotationTok}[1]{\textcolor[rgb]{0.56,0.35,0.01}{\textbf{\textit{#1}}}}
\newcommand{\AttributeTok}[1]{\textcolor[rgb]{0.77,0.63,0.00}{#1}}
\newcommand{\BaseNTok}[1]{\textcolor[rgb]{0.00,0.00,0.81}{#1}}
\newcommand{\BuiltInTok}[1]{#1}
\newcommand{\CharTok}[1]{\textcolor[rgb]{0.31,0.60,0.02}{#1}}
\newcommand{\CommentTok}[1]{\textcolor[rgb]{0.56,0.35,0.01}{\textit{#1}}}
\newcommand{\CommentVarTok}[1]{\textcolor[rgb]{0.56,0.35,0.01}{\textbf{\textit{#1}}}}
\newcommand{\ConstantTok}[1]{\textcolor[rgb]{0.00,0.00,0.00}{#1}}
\newcommand{\ControlFlowTok}[1]{\textcolor[rgb]{0.13,0.29,0.53}{\textbf{#1}}}
\newcommand{\DataTypeTok}[1]{\textcolor[rgb]{0.13,0.29,0.53}{#1}}
\newcommand{\DecValTok}[1]{\textcolor[rgb]{0.00,0.00,0.81}{#1}}
\newcommand{\DocumentationTok}[1]{\textcolor[rgb]{0.56,0.35,0.01}{\textbf{\textit{#1}}}}
\newcommand{\ErrorTok}[1]{\textcolor[rgb]{0.64,0.00,0.00}{\textbf{#1}}}
\newcommand{\ExtensionTok}[1]{#1}
\newcommand{\FloatTok}[1]{\textcolor[rgb]{0.00,0.00,0.81}{#1}}
\newcommand{\FunctionTok}[1]{\textcolor[rgb]{0.00,0.00,0.00}{#1}}
\newcommand{\ImportTok}[1]{#1}
\newcommand{\InformationTok}[1]{\textcolor[rgb]{0.56,0.35,0.01}{\textbf{\textit{#1}}}}
\newcommand{\KeywordTok}[1]{\textcolor[rgb]{0.13,0.29,0.53}{\textbf{#1}}}
\newcommand{\NormalTok}[1]{#1}
\newcommand{\OperatorTok}[1]{\textcolor[rgb]{0.81,0.36,0.00}{\textbf{#1}}}
\newcommand{\OtherTok}[1]{\textcolor[rgb]{0.56,0.35,0.01}{#1}}
\newcommand{\PreprocessorTok}[1]{\textcolor[rgb]{0.56,0.35,0.01}{\textit{#1}}}
\newcommand{\RegionMarkerTok}[1]{#1}
\newcommand{\SpecialCharTok}[1]{\textcolor[rgb]{0.00,0.00,0.00}{#1}}
\newcommand{\SpecialStringTok}[1]{\textcolor[rgb]{0.31,0.60,0.02}{#1}}
\newcommand{\StringTok}[1]{\textcolor[rgb]{0.31,0.60,0.02}{#1}}
\newcommand{\VariableTok}[1]{\textcolor[rgb]{0.00,0.00,0.00}{#1}}
\newcommand{\VerbatimStringTok}[1]{\textcolor[rgb]{0.31,0.60,0.02}{#1}}
\newcommand{\WarningTok}[1]{\textcolor[rgb]{0.56,0.35,0.01}{\textbf{\textit{#1}}}}
\usepackage{graphicx}
\makeatletter
\def\maxwidth{\ifdim\Gin@nat@width>\linewidth\linewidth\else\Gin@nat@width\fi}
\def\maxheight{\ifdim\Gin@nat@height>\textheight\textheight\else\Gin@nat@height\fi}
\makeatother
% Scale images if necessary, so that they will not overflow the page
% margins by default, and it is still possible to overwrite the defaults
% using explicit options in \includegraphics[width, height, ...]{}
\setkeys{Gin}{width=\maxwidth,height=\maxheight,keepaspectratio}
% Set default figure placement to htbp
\makeatletter
\def\fps@figure{htbp}
\makeatother
\setlength{\emergencystretch}{3em} % prevent overfull lines
\providecommand{\tightlist}{%
  \setlength{\itemsep}{0pt}\setlength{\parskip}{0pt}}
\setcounter{secnumdepth}{-\maxdimen} % remove section numbering
\ifLuaTeX
  \usepackage{selnolig}  % disable illegal ligatures
\fi
\IfFileExists{bookmark.sty}{\usepackage{bookmark}}{\usepackage{hyperref}}
\IfFileExists{xurl.sty}{\usepackage{xurl}}{} % add URL line breaks if available
\urlstyle{same} % disable monospaced font for URLs
\hypersetup{
  pdftitle={Processing data sources for analysis},
  pdfauthor={Michael-Philipp Stiebing},
  hidelinks,
  pdfcreator={LaTeX via pandoc}}

\title{Processing data sources for analysis}
\author{Michael-Philipp Stiebing}
\date{2023-03-01}

\begin{document}
\maketitle

\fontsize{10}{12}
\selectfont

\begin{itemize}
\tightlist
\item
  Start by importing libraries
\end{itemize}

\fontsize{9}{11}
\selectfont

\begin{Shaded}
\begin{Highlighting}[]
\FunctionTok{library}\NormalTok{(tidyverse)}
\FunctionTok{library}\NormalTok{(lubridate)}
\FunctionTok{library}\NormalTok{(ggplot2)}
\FunctionTok{library}\NormalTok{(readr)}
\end{Highlighting}
\end{Shaded}

\fontsize{10}{12}
\selectfont

\begin{itemize}
\tightlist
\item
  Read the csv files into separate dataframes
\item
  Combine all dataframes into a single dataframe
\end{itemize}

\fontsize{9}{11}
\selectfont

\begin{Shaded}
\begin{Highlighting}[]
\FunctionTok{setwd}\NormalTok{(}\StringTok{"/home/mikiR/remote\_transfer/"}\NormalTok{)}
\NormalTok{X202110\_divvy\_tripdata }\OtherTok{\textless{}{-}} 
  \FunctionTok{read\_csv}\NormalTok{(}\StringTok{"./20221205{-}capstone\_datascience{-}01/002\_data/001\_csv/202110{-}divvy{-}tripdata.csv"}\NormalTok{)}
\NormalTok{X202111\_divvy\_tripdata }\OtherTok{\textless{}{-}} 
  \FunctionTok{read\_csv}\NormalTok{(}\StringTok{"./20221205{-}capstone\_datascience{-}01/002\_data/001\_csv/202111{-}divvy{-}tripdata.csv"}\NormalTok{)}
\NormalTok{X202112\_divvy\_tripdata }\OtherTok{\textless{}{-}} 
  \FunctionTok{read\_csv}\NormalTok{(}\StringTok{"./20221205{-}capstone\_datascience{-}01/002\_data/001\_csv/202112{-}divvy{-}tripdata.csv"}\NormalTok{)}
\NormalTok{X202201\_divvy\_tripdata }\OtherTok{\textless{}{-}} 
  \FunctionTok{read\_csv}\NormalTok{(}\StringTok{"./20221205{-}capstone\_datascience{-}01/002\_data/001\_csv/202201{-}divvy{-}tripdata.csv"}\NormalTok{)}
\NormalTok{X202202\_divvy\_tripdata }\OtherTok{\textless{}{-}} 
  \FunctionTok{read\_csv}\NormalTok{(}\StringTok{"./20221205{-}capstone\_datascience{-}01/002\_data/001\_csv/202202{-}divvy{-}tripdata.csv"}\NormalTok{)}
\NormalTok{X202203\_divvy\_tripdata }\OtherTok{\textless{}{-}} 
  \FunctionTok{read\_csv}\NormalTok{(}\StringTok{"./20221205{-}capstone\_datascience{-}01/002\_data/001\_csv/202203{-}divvy{-}tripdata.csv"}\NormalTok{)}
\NormalTok{X202204\_divvy\_tripdata }\OtherTok{\textless{}{-}} 
  \FunctionTok{read\_csv}\NormalTok{(}\StringTok{"./20221205{-}capstone\_datascience{-}01/002\_data/001\_csv/202204{-}divvy{-}tripdata.csv"}\NormalTok{)}
\NormalTok{X202205\_divvy\_tripdata }\OtherTok{\textless{}{-}} 
  \FunctionTok{read\_csv}\NormalTok{(}\StringTok{"./20221205{-}capstone\_datascience{-}01/002\_data/001\_csv/202205{-}divvy{-}tripdata.csv"}\NormalTok{)}
\NormalTok{X202206\_divvy\_tripdata }\OtherTok{\textless{}{-}} 
  \FunctionTok{read\_csv}\NormalTok{(}\StringTok{"./20221205{-}capstone\_datascience{-}01/002\_data/001\_csv/202206{-}divvy{-}tripdata.csv"}\NormalTok{)}
\NormalTok{X202207\_divvy\_tripdata }\OtherTok{\textless{}{-}} 
  \FunctionTok{read\_csv}\NormalTok{(}\StringTok{"./20221205{-}capstone\_datascience{-}01/002\_data/001\_csv/202207{-}divvy{-}tripdata.csv"}\NormalTok{)}
\NormalTok{X202208\_divvy\_tripdata }\OtherTok{\textless{}{-}} 
  \FunctionTok{read\_csv}\NormalTok{(}\StringTok{"./20221205{-}capstone\_datascience{-}01/002\_data/001\_csv/202208{-}divvy{-}tripdata.csv"}\NormalTok{)}
\NormalTok{X202209\_divvy\_tripdata }\OtherTok{\textless{}{-}} 
  \FunctionTok{read\_csv}\NormalTok{(}\StringTok{"./20221205{-}capstone\_datascience{-}01/002\_data/001\_csv/202209{-}divvy{-}tripdata.csv"}\NormalTok{)}
\NormalTok{X202210\_divvy\_tripdata }\OtherTok{\textless{}{-}} 
  \FunctionTok{read\_csv}\NormalTok{(}\StringTok{"./20221205{-}capstone\_datascience{-}01/002\_data/001\_csv/202210{-}divvy{-}tripdata.csv"}\NormalTok{)}

\NormalTok{all\_trips }\OtherTok{\textless{}{-}} \FunctionTok{bind\_rows}\NormalTok{(}
\NormalTok{                  X202110\_divvy\_tripdata, X202111\_divvy\_tripdata, X202112\_divvy\_tripdata, X202201\_divvy\_tripdata,}
\NormalTok{                  X202202\_divvy\_tripdata, X202203\_divvy\_tripdata, X202204\_divvy\_tripdata, X202205\_divvy\_tripdata,}
\NormalTok{                  X202206\_divvy\_tripdata, X202207\_divvy\_tripdata, X202208\_divvy\_tripdata, X202209\_divvy\_tripdata, }
\NormalTok{                  X202210\_divvy\_tripdata)}
\end{Highlighting}
\end{Shaded}

\fontsize{10}{12}
\selectfont

\begin{itemize}
\tightlist
\item
  Create new columns from date field, split into

  \begin{itemize}
  \tightlist
  \item
    month
  \item
    day
  \item
    year
  \item
    day of week
  \end{itemize}
\end{itemize}

\fontsize{9}{11}
\selectfont

\begin{Shaded}
\begin{Highlighting}[]
\NormalTok{all\_trips}\SpecialCharTok{$}\NormalTok{date }\OtherTok{\textless{}{-}} \FunctionTok{as.Date}\NormalTok{(all\_trips}\SpecialCharTok{$}\NormalTok{started\_at)}
\NormalTok{all\_trips}\SpecialCharTok{$}\NormalTok{month }\OtherTok{\textless{}{-}} \FunctionTok{format}\NormalTok{(}\FunctionTok{as.Date}\NormalTok{(all\_trips}\SpecialCharTok{$}\NormalTok{date), }\StringTok{"\%m"}\NormalTok{)}
\NormalTok{all\_trips}\SpecialCharTok{$}\NormalTok{day }\OtherTok{\textless{}{-}} \FunctionTok{format}\NormalTok{(}\FunctionTok{as.Date}\NormalTok{(all\_trips}\SpecialCharTok{$}\NormalTok{date), }\StringTok{"\%d"}\NormalTok{)}
\NormalTok{all\_trips}\SpecialCharTok{$}\NormalTok{year }\OtherTok{\textless{}{-}} \FunctionTok{format}\NormalTok{(}\FunctionTok{as.Date}\NormalTok{(all\_trips}\SpecialCharTok{$}\NormalTok{date), }\StringTok{"\%Y"}\NormalTok{)}
\NormalTok{all\_trips}\SpecialCharTok{$}\NormalTok{day\_of\_week }\OtherTok{\textless{}{-}} \FunctionTok{format}\NormalTok{(}\FunctionTok{as.Date}\NormalTok{(all\_trips}\SpecialCharTok{$}\NormalTok{date), }\StringTok{"\%A"}\NormalTok{)}
\end{Highlighting}
\end{Shaded}

\fontsize{10}{12}
\selectfont

\begin{itemize}
\tightlist
\item
  Create column ridelength calculated from the difference of ended\_at
  and started\_at
\item
  Convert column value into numeric
\end{itemize}

\fontsize{9}{11}
\selectfont

\begin{Shaded}
\begin{Highlighting}[]
\NormalTok{all\_trips}\SpecialCharTok{$}\NormalTok{ride\_length }\OtherTok{\textless{}{-}} \FunctionTok{difftime}\NormalTok{(all\_trips}\SpecialCharTok{$}\NormalTok{ended\_at,all\_trips}\SpecialCharTok{$}\NormalTok{started\_at)}
\NormalTok{all\_trips}\SpecialCharTok{$}\NormalTok{ride\_length }\OtherTok{\textless{}{-}} \FunctionTok{as.numeric}\NormalTok{(}\FunctionTok{as.character}\NormalTok{(all\_trips}\SpecialCharTok{$}\NormalTok{ride\_length))}
\end{Highlighting}
\end{Shaded}

\fontsize{10}{12}
\selectfont

\begin{itemize}
\tightlist
\item
  Remove bad data
\end{itemize}

\fontsize{9}{11}
\selectfont

\begin{Shaded}
\begin{Highlighting}[]
\NormalTok{all\_trips\_v2 }\OtherTok{\textless{}{-}}\NormalTok{ all\_trips[}\SpecialCharTok{!}\NormalTok{(all\_trips}\SpecialCharTok{$}\NormalTok{ride\_length}\SpecialCharTok{\textless{}}\DecValTok{0}\NormalTok{),]}
\end{Highlighting}
\end{Shaded}

\fontsize{10}{12}
\selectfont

\begin{itemize}
\tightlist
\item
  Move bad data into a dataframe to doublecheck
\end{itemize}

\fontsize{9}{11}
\selectfont

\begin{Shaded}
\begin{Highlighting}[]
\NormalTok{all\_trips\_errors }\OtherTok{\textless{}{-}}\NormalTok{ all\_trips[(all\_trips}\SpecialCharTok{$}\NormalTok{ride\_length}\SpecialCharTok{\textless{}}\DecValTok{0}\NormalTok{),]}
\end{Highlighting}
\end{Shaded}

\fontsize{10}{12}
\selectfont

\begin{itemize}
\tightlist
\item
  Trying to calculate the distance between gps coordinates start\_lng \&
  start\_lat / end\_lng \& end\_lat
\item
  Import geosphere library
\end{itemize}

\fontsize{9}{11}
\selectfont

\begin{Shaded}
\begin{Highlighting}[]
\FunctionTok{library}\NormalTok{(geosphere) }
\end{Highlighting}
\end{Shaded}

\fontsize{10}{12}
\selectfont

\begin{itemize}
\tightlist
\item
  Calculate the distance between the start and end coordinates using the
  Haversine formula
\item
  Create new column geodist with the calculated distance
\end{itemize}

\fontsize{9}{11}
\selectfont

\begin{Shaded}
\begin{Highlighting}[]
\NormalTok{all\_trips\_v3 }\OtherTok{\textless{}{-}}\NormalTok{ all\_trips\_v2 }\SpecialCharTok{\%\textgreater{}\%} 
              \FunctionTok{mutate}\NormalTok{(}\AttributeTok{geodist =} \FunctionTok{distHaversine}\NormalTok{(}
                    \FunctionTok{cbind}\NormalTok{(all\_trips\_v2}\SpecialCharTok{$}\NormalTok{start\_lng,all\_trips\_v2}\SpecialCharTok{$}\NormalTok{start\_lat), }
                    \FunctionTok{cbind}\NormalTok{(all\_trips\_v2}\SpecialCharTok{$}\NormalTok{end\_lng,all\_trips\_v2}\SpecialCharTok{$}\NormalTok{end\_lat)))}
\end{Highlighting}
\end{Shaded}

\fontsize{10}{12}
\selectfont

\begin{itemize}
\tightlist
\item
  Filter all trips with distance = 0 into a dataframe called
  round\_trips, the assumtion being that when the trip ends where it
  started
\end{itemize}

\fontsize{9}{11}
\selectfont

\begin{Shaded}
\begin{Highlighting}[]
\NormalTok{round\_trips }\OtherTok{=} \FunctionTok{filter}\NormalTok{(all\_trips\_v3,geodist }\SpecialCharTok{==} \DecValTok{0}\NormalTok{)}
\end{Highlighting}
\end{Shaded}


\end{document}
